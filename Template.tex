\documentclass{article} % Sets up formatting rules for the document. Article is a basic style used everywhere

\usepackage{amsmath} % For math stuff
\usepackage{amssymb} % For math stuff
\usepackage{pifont} % For math stuff
\usepackage{graphicx} % For inputting pictures
\graphicspath{{./images/}} % Set the location for your images. You should have a subfolder called images to keep the project clean
\usepackage[letterpaper, portrait, margin=1in]{geometry} % Sets up some margins. The defaults aren't good imo
\usepackage{authblk} % Lets us use a cool trick for making multiline author blocks. Useful for when we have to write our names and IDs. Gets too much for 1 line

\begin{titlepage} % Lets you define a cover page. You could probably add more stuff here if you want
    \title{CLASS NAME ASSIGNMENT NAME}
    
    % Fill up the \author block with your names and ids, separated by \and commands (for it to be multiline)
    \author[1]{}
    
\end{titlepage}

\begin{document} % Starts the document and applies the article style to it
    
    \maketitle % Makes the title page based on what you put above

    \newpage % You should enforce a new page after the title page

    \section*{Problem 1}
    
    \paragraph{Can do math inline: $2 + 2 = 4$ }

    \section*{Problem 2} % We can have sections...

        \subsection*{1.} % We can have subsections...

        \paragraph{Can also do it in block form:} % It's good to write in paragraph (or other) environments for some formatting rules to be autmatically applied. Avoids warnings

        \begin{align*}
            1+1+1+1 & = 4 \\
            2+2 & = 4 \\
            4 & = 4 \\
        \end{align*}

        \subsection*{2.}

        \paragraph{or with a label to the side:}

        \begin{align}
            \begin{split}
                1 + 2 &= 3\\
                1 &= 3 - 2
            \end{split}
          \end{align}

        \subsection*{3.}

        \begin{equation} \label{eq1}
            \begin{split}
            A & = \frac{\pi r^2}{2} \\
             & = \frac{1}{2} \pi r^2
            \end{split}
            \end{equation}

    \section*{Problem 3}

    \section*{Problem 4}

        \subsection*{1.}

        \subsection*{2.}

            \subsubsection*{(a)} % We can have subsubsections...

            \paragraph{}

            \subsubsection*{(b)}

            \subsubsection*{(c)}

    \section*{Problem 5}

    % Here's a generic table. Put stuff between the & symbols. The & are for proper alignment btw
    \begin{center}
        \begin{tabular}{ |l|c|c|c|c|c| } 
         \hline
          &  &  &  &  &  \\ % This should be the column headings. Row headings are before the first & of each row
         \hline
          &  &  &  &  & \\ 
         \hline
          &  &  &  &  &  \\ 
         \hline
            &  &  &  &  &  \\ 
         \hline
        \end{tabular}
    \end{center}

    \section*{Problem 6}
        % Here's a cheap, easy way to put in pictures. Make sure to properly include your graphics path at the top.
        % It's commented to avoid errors because we have no pictures now

        % \begin{center}
        %     \includegraphics[scale=0.5]{hasse}
        % \end{center}

    \section*{Problem 7}
    

    \section*{Problem 8}

    \section*{Problem 9}


\end{document} % Ends the document